\newpage
\begin{thebibliography}{}

\bibitem{Jiménez 2023}
Jiménez, A. (2023). \textit{Programación imperativa: conceptos y ejemplos prácticos.}. elblogpython.com Obtenido de: \url{https://elblogpython.com/informatica/programacion-imperativa-conceptos-y-ejemplos-practicos/}
‌
\bibitem{Jonathan2021}
Jonathan B. ,Baltazar, M.  (2021). \textit{Características de la POO}. portalacademico.cch.unam.mx. Obtenido de: \url{https://portalacademico.cch.unam.mx/cibernetica1/algoritmos-y-codificacion/caracteristicas-POO#:~:text=El%20paradigma%20de%20la%20programaci%C3%B3n,%2C%20herencia%2C%20encapsulamiento%20y%20polimorfismo.}

\bibitem{maria2023}
María C. (2023). \textit{Qué es Java, para qué sirve, características e historia.}. hubspot.es. Obtenido de: \url{https://blog.hubspot.es/website/que-es-java#historia}
‌

\bibitem{jose2023}
Jose S. (2004). \textit{ Qué es lenguaje C: el origen, las ventajas, las características y la sintaxis del lenguaje de programación.}. ebac.mx. Obtenido de: \url{https://ebac.mx/blog/que-es-lenguaje-c#title1}
‌
‌
\bibitem{uma2004}
UMA. (2004). \textit{LECCION 3. FUNCIONES (I)}. Uma.es. Obtenido de: \url{http://www.lcc.uma.es/~iaic/LISP3.pdf}

\bibitem{geeks2021}
GeeksforGeeks. (2021). \textit{Lambda functions in LISP}. GeeksforGeeks. Obtenido de: \url{https://www.geeksforgeeks.org/lambda-functions-in-lisp/}

\bibitem{greyrat2022}
Greyrat, R. (2022). \textit{Funciones Lambda en LISP}. Barcelonageeks.com. Obtenido de: \url{https://barcelonageeks.com/funciones-lambda-en-lisp/}

\bibitem{UNAL2024}
UNAL. (2024). \textit{Programación lógica UNAL.}. github.io. Obtenido de: \url{https://ferestrepoca.github.io/paradigmas-de-programacion/proglogica/logica_teoria/proglogica.html}
‌
\bibitem{pulido2021}
Pulido, K. R., Soto, M. R., \& Mendoza, J. E. (2021). \textit{Conceptos Generales}. Google Docs. Obtenido de: \url{https://drive.google.com/file/d/12JEvZQh8F8UGC7Jep3IUZSchagBBa-Yd/view}


‌\bibitem{llp2023}
LLP. (2023). \textit{Tema 2: Programación fucional}. 
Github.io. Obtenido de:
\url{https://domingogallardo.github.io/apuntes-lpp/teoria/tema02-programacion-funcional/tema02-programacion-funcional.html}

\bibitem{Juandec2021}
Juandc. (2021). \textit{Paradigmas de programación: ¿qué son y cuál aprender?}. Platzi. Obtenido de: \url{https://platzi.com/blog/paradigmas-programacion/}

\end{thebibliography}

