\section{Requerimientos del Sistema}

\subsection{Requerimientos Funcionales}

\subsubsection{RF1: Autenticación de Usuarios}
\begin{itemize}
    \item El sistema permitirá el acceso mediante credenciales únicas (correo electrónico y contraseña).
    \item Se implementarán dos niveles de acceso:
    \begin{itemize}
        \item \textbf{Usuarios regulares:} Pueden reportar incidentes, actualizar estados y validar resoluciones.
        \item \textbf{Administradores:} Tienen acceso a funciones administrativas y gestión de usuarios.
    \end{itemize}
    \item Las contraseñas deberán cumplir con requisitos mínimos de seguridad:
    \begin{itemize}
        \item Mínimo 8 caracteres.
        \item Al menos una letra mayúscula.
        \item Al menos un número.
        \item Al menos un carácter especial.
    \end{itemize}
    \item El sistema bloqueará la cuenta temporalmente después de 3 intentos fallidos de inicio de sesión.
    \item El sistema mantendrá un registro de todos los intentos de inicio de sesión para fines de auditoría.
\end{itemize}

\subsubsection{RF2: Gestión de Usuarios}
\begin{itemize}
    \item El sistema permitirá el registro, edición y eliminación de usuarios.
    \item Los usuarios podrán iniciar sesión de forma segura.
    \item Los usuarios podrán editar su perfil (Alias y Contraseña).
\end{itemize}

\subsubsection{RF3: Registro de Incidentes}
\begin{itemize}
    \item Los usuarios podrán marcar la ubicación exacta del incidente en un mapa.
    \item Se permitirá subir hasta 5 fotografías por incidente.
    \item Los usuarios deberán categorizar el tipo de incidente (baches y desperfectos, luminaria, accidentes vehiculares, obstáculos).
    \item Se requerirá una descripción detallada del problema (limite 500 caracteres).
    \item El sistema registrará automáticamente la fecha y hora del reporte.
\end{itemize}

\subsubsection{RF4: Actualización de Incidentes}
\begin{itemize}
    \item Cualquier usuario podrá actualizar el estado de un incidente.
    \item Se requerirá evidencia fotográfica para marcar un incidente como resuelto.
    \item Se mantendrá un historial de actualizaciones.
    \item Se notificará al creador del reporte sobre las actualizaciones.
\end{itemize}

\subsubsection{RF5: Visualización de Incidentes}
\begin{itemize}
    \item Mostrar incidentes en un mapa interactivo.
    \item Filtrar incidentes por categoría, estado y fecha.
    \item Permitir búsqueda por dirección o coordenadas.
\end{itemize}

\subsubsection{RF6: Validación Comunitaria de Resolución}
\begin{itemize}
    \item Cuando un usuario marca un incidente como resuelto, este no se actualizará automáticamente.
    \item Otros usuarios podrán confirmar o disputar la resolución.
    \item Si hay disputas, el incidente permanecerá "en revisión" hasta que un administrador lo valide.
    \item Se notificará a los usuarios involucrados sobre el resultado de la validación.
\end{itemize}


\subsection{Requerimientos No Funcionales}

\subsubsection{RNF1: Seguridad}
\begin{itemize}
    \item Implementación de autenticación.
    \item Uso de autenticación segura para evitar modificaciones no autorizadas.
    \item Encriptación de datos sensibles (como contraseñas).
    \item Validación de inputs para prevenir inyecciones.
    \item Control de acceso basado en roles.
    \item Política de grupo:
    \begin{itemize}
        \item No todos los grupos tienen acceso a la base de datos.
        \item Solo usuarios registrados pueden registrar incidentes.
        \item Solo administradores pueden eliminar cualquier incidente.
        \item Solo el usuario dueño del incidente puede borrarlo.
    \end{itemize}
\end{itemize}

\subsubsection{RNF2: Rendimiento}
\begin{itemize}
    \item Tiempo de respuesta máximo de 3 segundos.
    \item Soporte para 100 usuarios simultáneos.
    \item Optimización de imágenes automática.
\end{itemize}

\subsubsection{RNF3: Disponibilidad}
\begin{itemize}
    \item El sistema deberá estar disponible al menos el 99\% del tiempo.
    \item Recuperación ante desastres.
\end{itemize}

\subsubsection{RNF4: Usabilidad}
\begin{itemize}
    \item No se puede modificar un incidente sin pruebas.
    \item Diseño intuitivo y accesible.
    \item Navegación simple con número reducido de pasos para completar tareas.
\end{itemize}
