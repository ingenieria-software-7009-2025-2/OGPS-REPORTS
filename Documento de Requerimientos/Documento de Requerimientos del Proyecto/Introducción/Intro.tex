\section{Introducción}

\subsection{Propósito del Sistema}
El sistema tiene como propósito principal facilitar la participación ciudadana en la identificación y seguimiento de problemas urbanos mediante una plataforma web colaborativa (OGPS-Reports). Esta herramienta permitirá a los ciudadanos reportar, documentar y dar seguimiento a incidentes urbanos como baches, luminarias descompuestas y obstáculos en la vía pública.

\subsection{Alcance}
El sistema abarcará:
\begin{itemize}
    \item Registro y gestión de incidentes urbanos con geolocalización.
    \item Documentación fotográfica de incidentes.
    \item Seguimiento del estado de los incidentes.
    \item Visualización interactiva de incidentes en un mapa.
    \item Validación comunitaria de la resolución de incidentes mediante evidencia fotográfica.
\end{itemize}

\subsection{Definiciones y Abreviaciones}
\begin{itemize}
    \item \textbf{OGPS-Reports:} Online Geospatial Public System - Reports
    \item \textbf{Incidente:} Cualquier problema en la infraestructura urbana que requiera atención.
    \item \textbf{Usuario:} Persona que puede acceder al sistema (puede ser usuario regular o administrador).
    \item \textbf{Administrador:} Usuario con privilegios especiales que puede gestionar reportes, validar o rechazar actualizaciones de estado. También puede gestionar contenido inapropiado y supervisar la calidad de los reportes.
    \item \textbf{Mapa Interactivo:} Representación visual que permite la geolocalización de incidentes.
    \item \textbf{Estado del Incidente:} Representa el progreso de resolución del problema (\textit{Reportado, En proceso, Resuelto}).
    \item \textbf{Prueba:} Una o varias fotos del accidente reportado por el usuario. 
\end{itemize}
